% main.tex
% Universitaire Papers - Full Documentation & Usage Guide
% Author: Matt
% Purpose: A self-contained manual that explains paperSettings.tex,
%          standardCommands.tex and demonstrates how to use the macros.
%
% Save this file as main.tex in the root of the repo alongside:
%   - paperSettings.tex
%   - standardCommands.tex
%   - main.bib
%
% Compile with: buildLatexDocument.sh (or fullBuild.sh)
% Or upload to Overleaf and compile there.
% -----------------------------------------

% ----------- Document settings -----------
\documentclass[nonacm, sigconf, balance=true]{acmart}
\include{core/paperSettings} % import custom general paper settings
\include{core/standardCommands} % import custom commands and packages
% -----------------------------------------
\begin{document}

    \begin{SimpleTable}[s{1}s{.3}s{.3}]{}{}
        \TableHeader{TERM & LATEX & LATEX COMMAND}
        \TableRow{1. there exists at least one & \exists & \verb|\exists|}
        \TableRow{2. there exists one and only one & \exists! & \verb|\exists!|}
        \TableRow{3. there is no & \nexists & \verb|\nexists|}
        \TableRow{4. for all & \forall & \verb|\forall|}
        \TableRow{5. not (logical not) & \neg & \verb|\neg|}
        \TableRow{6. or (logical or) & \lor & \verb|\lor|}
        \TableRow{7. division & \div & \verb|\div|}
        \TableRow{8. and (logical and) & \land & \verb|\land|}
        \TableRow{9. implies & \implies & \verb|\implies|}
        \TableRow{10. right implication & \Rightarrow & \verb|\Rightarrow|}
        \TableRow{11. is implied by (only if) & \Longleftarrow & \verb|\Longleftarrow|}
        \TableRow{12. left implication & \Leftarrow & \verb|\Leftarrow|}
        \TableRow{13. if and only if, iff & \iff & \verb|\iff|}
        \TableRow{14. equivalence & \Leftrightarrow & \verb|\Leftrightarrow|}
        \TableRow{15.Subset & \subset & \verb|\subset|}
        \TableRow{16. Logical XOR (exclusive or) & \oplus & \verb|\oplus|}
        \TableRow{17. Union of sets & \cup & \verb|\cup|}
        \TableRow{18. Empty set & \emptyset & \verb|\emptyset|}
        \TableRow{19. Intersection of sets & \cap & \verb|\cap|}
        \TableRow{20. Union of sets & \cup & \verb|\cup|}
    \end{SimpleTable}

    \clearpage
    \onecolumn

    \section{Propositielogica}
    Propositielogica is de studie van bewijzen

    \subsection{Proposities}
    We gebruiken in logica variabelen zoals P en Q, dit zijn \textit{atomisch} proposities, deze zijn altijd geschreven in HOOFDLETTERS.
    Deze zijn niet verder op te delen, dus niet opgebouwd van kleinere delen zoals implicaties etc. deze waardes zijn Waar/True/1 of Onwaar/False/0.

    Je hebt ook kleine letters p, q, \ldots, dit zijn niet-atomische proposities, het zijn geen propositionele formules, maar eerder \textit{metavariabelen}.

    \begin{multicols}{2}

    Elke propositionele formule bestaat uit:
    \begin{itemize}
        \item Atomitsche proposities (P, Q, R, \ldots)
        \item true, (T, Waar, 1)
        \item false, (F, Onwaar, 0)
    \end{itemize}

    Deze kunnen ook kleiner opgedeeld zijn, dan zijn dit ook propositionele formules:
    \begin{itemize}
        \item $P \implies Q$ - \textbf{implicatie} als P, dan Q
        \item $P \land Q$ - \textbf{conjunctie} P én Q
        \item $P \lor Q$ - \textbf{disjunctie} P of Q
        \item $\neg P$ - \textbf{negatie} (niet P / P houd geen stand)
    \end{itemize}

    \end{multicols}

    Zoals in de wiskunde ook is heeft propositielogica óók een volgorde, deze is:
    \begin{enumerate}
        \item Haakjes ()
        \item Negatie $\neg$
        \item Conjunctie $\land$
        \item Disjunctie $\lor$
        \item Implicatie $\implies$
    \end{enumerate}

    $\neg P \lor Q \implies Q \land P$ moet gelezen worden als $((\neg P) \lor Q) \implies (Q \land P)$

    Om dit te onthouden kan je deze zij onthouden: ``Hoe Navigeert Connie De Ijssel''

    \subsection{Semantiek van propositie operatoren}

    \begin{center}
% ======== Row 1 ========
        \begin{minipage}[t]{0.32\textwidth}
            \begin{SimpleTable}[s{.3}s{.3}]{Truthtable van negatie/NOT}{}
                \TableHeader{$P$ & $\neg P$}
                \TableRow{0 & 1}
                \TableRow{1 & 0}
            \end{SimpleTable}
        \end{minipage}
        \hfill
        \begin{minipage}[t]{0.32\textwidth}
            \begin{SimpleTable}[s{.3}s{.3}s{.3}]{Truthtable van Conjunctie/AND}{}
                \TableHeader{$P$ & $Q$ & $P \land Q$}
                \TableRow{0 & 0 & 0}
                \TableRow{0 & 1 & 0}
                \TableRow{1 & 0 & 0}
                \TableRow{1 & 1 & 1}
            \end{SimpleTable}
        \end{minipage}
        \hfill
        \begin{minipage}[t]{0.32\textwidth}
            \begin{SimpleTable}[s{.3}s{.3}s{.3}]{Truthtable van Disjunctie/OR}{}
                \TableHeader{$P$ & $Q$ & $P \lor Q$}
                \TableRow{0 & 0 & 0}
                \TableRow{0 & 1 & 1}
                \TableRow{1 & 0 & 1}
                \TableRow{1 & 1 & 1}
            \end{SimpleTable}
        \end{minipage}

        \vspace{1em}

% ======== Row 2 ========
        \begin{minipage}[t]{0.32\textwidth}
            \begin{SimpleTable}[s{.3}s{.3}s{.3}]{Truthtable van XOR}{}
                \TableHeader{$P$ & $Q$ & $P \oplus Q$}
                \TableRow{0 & 0 & 0}
                \TableRow{0 & 1 & 1}
                \TableRow{1 & 0 & 1}
                \TableRow{1 & 1 & 0}
            \end{SimpleTable}
        \end{minipage}
        \hfill
        \begin{minipage}[t]{0.32\textwidth}
            \begin{SimpleTable}[s{.3}s{.3}s{.3}]{Truthtable van implicatie}{}
                \TableHeader{$P$ & $Q$ & $P \implies Q$}
                \TableRow{0 & 0 & 1}
                \TableRow{0 & 1 & 1}
                \TableRow{1 & 0 & 0}
                \TableRow{1 & 1 & 1}
            \end{SimpleTable}
        \end{minipage}
        \hfill
        \begin{minipage}[t]{0.32\textwidth}
            \begin{SimpleTable}[s{.3}s{.3}s{.3}]{Truthtable van equivalent/==}{}
                \TableHeader{$P$ & $Q$ & $P \equiv Q$}
                \TableRow{0 & 0 & 1}
                \TableRow{0 & 1 & 0}
                \TableRow{1 & 0 & 0}
                \TableRow{1 & 1 & 1}
            \end{SimpleTable}
        \end{minipage}
    \end{center}

    Maar stel, je wilt iets moeilijks bewijzen zoals $\neg P \lor Q \implies Q \land P$, dan kan je dat op de volgende manier doen:

    \vsmall

    \begin{SimpleTable}[s{.3}s{.3}s{.3}s{.3}s{.3}s{.3}]{Truthtable van moeilijkere propositie formule}{}
        \TableHeader{$P$ & $Q$ & $\neg P$ & $\neg P \lor Q$ & $ Q \land P$ &  $\neg P \lor Q \implies Q \land P$}
        \TableRow{0 & 0 & 1 & 1 & 0 & 0}
        \TableRow{0 & 1 & 1 & 1 & 0 & 0}
        \TableRow{1 & 0 & 0 & 0 & 0 & 1}
        \TableRow{1 & 1 & 0 & 1 & 1 & 1}
    \end{SimpleTable}

    Je kan ook met verschillende kleuren pennen een kleine thruthtable schrijven, maar dit is een hele nette manier om het ook te doen.

    \subsection{Voorbij thruth tables}
    Je kan d.m.v een truth table bewijzen of een formule wel of niet houd, maar dat kan ook anders, bijvoorbeeld door het versimpelen van de formules.

    \begin{multicols}{2}

        \begin{SimpleTable}[s{.1}s{.5}s{.4}]{}{}
            \TableHeader{ & Expression & Law}
            \TableRow{                  & $(p \implies q) \lor (q \implies p)$      & original}
            \TableRow{$\Leftrightarrow$ & $(\neg p \lor q) \lor (\neg q \lor p)$    & implication}
            \TableRow{$\Leftrightarrow$ & $\neg p \lor ((q \lor \neg q) \lor p)$    & associativity / rearrangement}
            \TableRow{$\Leftrightarrow$ & $\neg p \lor (T \lor p)$                  & tertium non datur}
            \TableRow{$\Leftrightarrow$ & $\neg p \lor T$                           & absorbing property of \(T\)}
            \TableRow{$\Leftrightarrow$ & $T$                                       & absorbing property of \(T\)}
        \end{SimpleTable}

        Deze afleiding kan je maken door de regels te gebruiken die hieronder stan geschreven (er zijn er nog meer).

    \end{multicols}

    \begin{multicols}{3}

    Commutativity:
    \begin{itemize}
        \item $p \land q \Leftrightarrow q \land p$
        \item $p \lor q \Leftrightarrow q \lor p$
    \end{itemize}

    Associativity:
    \begin{itemize}
        \item $p \land (q \land r) \Leftrightarrow (p \land q) \land r$
        \item $p \lor (q \lor r) \Leftrightarrow (p \lor q) \lor r$
    \end{itemize}

    Tertium non datur:
    \begin{itemize}
        \item $p \lor \neg p \Leftrightarrow T$
    \end{itemize}

    Idempotence:
    \begin{itemize}
        \item $p \land p \Leftrightarrow p$
        \item $p \lor p \Leftrightarrow p$
    \end{itemize}

    De Morgan:
    \begin{itemize}
        \item $\neg (p \lor q) \Leftrightarrow \neg p \land \neg q$
        \item $\neg (p \land q) \Leftrightarrow \neg p \lor \neg q$
    \end{itemize}

    Double negation (niet niet is wel):
    \begin{itemize}
        \item $p \Leftrightarrow \neg(\neg p)$
    \end{itemize}

    Properties of T en F:
    \begin{itemize}
        \item $p \lor F \Leftrightarrow p$
        \item $p \land F \Leftrightarrow F$
        \item $q \lor T \Leftrightarrow T$
        \item $q \land T \Leftrightarrow q$
    \end{itemize}

    Implication:
    \begin{itemize}
        \item $(p \implies q) \Leftrightarrow (\neg p \lor q)$
    \end{itemize}

    Contraposition:
    \begin{itemize}
        \item $(p \implies q) \Leftrightarrow (\neg q \implies \neg p)$
    \end{itemize}

        \end{multicols}

    %TODO: powerpoihnt vanaf hier https://uu.brightspace.com/content/enforced/44250-BETA--2025--2--INFOB1LI--V/slides/02-slides.pdf#page=90.98

\end{document}