% main.tex
% Universitaire Papers - Full Documentation & Usage Guide
% Author: Matt
% Purpose: A self-contained manual that explains paperSettings.tex,
%          standardCommands.tex and demonstrates how to use the macros.
%
% Save this file as main.tex in the root of the repo alongside:
%   - paperSettings.tex
%   - standardCommands.tex
%   - main.bib
%
% Compile with: buildLatexDocument.sh (or fullBuild.sh)
% Or upload to Overleaf and compile there.
% -----------------------------------------

% ----------- Document settings -----------
\documentclass[nonacm, sigconf, balance=true]{acmart}
\include{core/paperSettings} % import custom general paper settings
\include{core/standardCommands} % import custom commands and packages
% -----------------------------------------
\begin{document}

    \clearpage
    \onecolumn

% ====================================================
    \section[Semiotiek: \textbf{Mens-Informatie}]{Semiotiek}
    Semiotiek is het zelfde als 'tekenleer'. Denk hierbij aan:
    \begin{itemize}
        \item letters (A-Z)
        \item Karakters (chinees alfabet)
        \item Woorden
        \item Morsetekens/braille
        \item verkeerdborden, pictogrammern
        \item gebaren
        \item voorwerpen (witte vlag, e.d)
    \end{itemize}

    Semiotiek houd zich bezeig met elke activiteit, handeling of preces waarbij tekens worden gebruitk.
    Een teken defineren we als: "Alles dat een booschap communiceert van de zender naar de ontvanger"

    \subsection{De semiotische ladder}

    \subsection{Fysieke tekens (Technische laag)}
    Dit houd alles in van Klanken die je met je mond maakt tot gebaren, letters, Geuren e.d.
    Tekens kunnen op zichzelf al een betekenis hebben, maar kunnen ook met elkaar een betekenis hebben.

    \subsection{Het empirische niveau (Technische laag)}
    De manier waarop alle tekens (theoretisch) kunnen worden gecombineerd, geobserveerd of verstoord.

    % TODO: create a voorbeeld enviroment, that is an indented square box with some basic text at the top, and where you can write an explaination of what im telling

    Bijvoorbeeld: een systeem dat vier tekens kan produceren: a,b,c,d. Dan kan je op empirisch niveau analyseren hoe vaak elk van die tekens voorkomt (18,91\% van het NL lettergebruik is 'e')

    Zo kan je analyseren of een reeks tekens zinvol lijtk te zijn door of de hoeveelheid bepaalde letters "Nederlands" is.
    Of de hoeveelheid ruis op de lijn de communicatie niet in de weg zit.
    Welke waardes de letters voor Scrabble in verschillende talen zou moeten hebben (in NL is 'e' laag, 'Y' hoog bv)

    \subsection{Syntax}
    Syntax betekend kort gezegd, welke reeks tekens is geldig en welke niet.

    \begin{itemize}
        \item \textbf{Morfologie} is de leer van woordstructuur e de woordvorming. Het houd zich bezig met morfemen\footnote{boekenkast kan je opsplitesen in de morfemen "boek" en "kast", deze kan je niet verder opslitsen}
        \item \textbf{Syntaxis} bestaat uit woordontleding en zinsontleding
    \end{itemize}

    Moderne theorieën over syntax ijn sterk beïnvloed door de generatieve grammatica van de Amerikaans taalkundige Noam Chomsky.
    Voor het Nederlands gelden bijvoorbeeld deze \texit{contextvrije herschrijvregels}.

    \begin{multicols}{2}
        \begin{itemize}
            \item Zin → \textit{NC VC}
            \item VC → \textit{PC V}
            \item PC → \textit{P NC}
            \item NC → \textit{lidwoord N}
        \end{itemize}
        \columnbreak
        \begin{itemize}
            \item NC → \textit{noun phrase (de auto)}
            \item VC → \textit{verb phrase (gaat in de morgen rijden)}
            \item PC → \textit{prepositional phrase (in de morgen)}
            \item V → \textit{verb (rijden)}
            \item P → \textit{preposition (in)}
            \item N → \textit{noun (auto)}
        \end{itemize}
    \end{multicols}

    \subsection{Semantiek}
    Semantiek houdt zich bezig met de \textbf{betekenis} van woorden, zinnen, langere teksten of andere talige uitingen.
    Die betekenis kan bijvoorbeeld een defenties in de Van Dale zijn, jouw kennis van een bepaald onderwerp, de persoon/personen die jij met het onderwerp associeert met de naam Roderick.
    Dit is dus het niveau waarbij de propositie Roderick een betekenis in de "echte wereld" krijgt.

    \textbf{Objectivisme}: de vertaling van een syntactische structuur naar de fysieke wereld is voor iedereen hetzelfde.
    \newline → \textit{De wereld is objectief te kennen.}

    \textbf{Constructivisme}: Legt nadruk of het feit dat kennis tot stand komt door een actieve constructie, eerder dan een passieve representatie vand e werkelijkheid.
    \newline → \textit{Wij creeëren, passen aan (, of vergeten) de defentieis van woorden of zinnen altijd, vooral waarneer er een spraakverwarring of misverstand kan ontstaan.}

    \subsection{pragmatiek}
    Doorgaans wordt een bepaalde communicatie-uiting gedaan een bepaalde bedoeling, een bepaalde \textit{intentie}
    We verwachten dan dat de andere pratijen (mensen of systemen) reageren zoals wij het bedoeld hebben:

    \begin{itemize}
        \item Dat iemand het licht uitschakelt als je zegt dat het licht nog aan is.
        \item Dat een gesprek wordt afgerond als jij demonstratief telkens op je horloge kijkt.
        \item Dat een zoekmachine je meest relevante, informatieve website voor de zoekopdracht toont
        \item Dat diezelfde zoekmachine je direct nieuws, defenities en statistieken toont voor een query
    \end{itemize}


    \section{kennisbanken}
    Het is een manier van beheer van 'kennis'. Denk aan kennis die in een bepaald bedrijf zit en opgeslagen moet worden.
    Je hebt hiervoor relationele databases, verschillende tabellen in een database kunnen aan eklaar gekoppeld worden dmv id's e.d.
    Deze tabellen hebben allemaal een unieke key.

\end{document}