% main.tex
% Universitaire Papers - Full Documentation & Usage Guide
% Author: Matt
% Purpose: A self-contained manual that explains paperSettings.tex,
%          standardCommands.tex and demonstrates how to use the macros.
%
% Save this file as main.tex in the root of the repo alongside:
%   - paperSettings.tex
%   - standardCommands.tex
%   - main.bib
%
% Compile with: buildLatexDocument.sh (or fullBuild.sh)
% Or upload to Overleaf and compile there.
% -----------------------------------------

% ----------- Document settings -----------
\documentclass[nonacm, sigconf, balance=true]{acmart}
\include{core/paperSettings} % import custom general paper settings
\include{core/standardCommands} % import custom commands and packages

\def\documentTitle      {Informatie Uitwisseling Visualisatie Opdracht} % replace with actual document title
\def\documentSubtitle   {} % replace with actual subtitle if any, else leave empty
\def\authorName         {Matt ter Steege} % replace with actual author name
\def\authorEmail        {m.j.ter.steege@students.uu.nl} % replace with actual author email
\def\institutionName    {Universiteit Utrecht} % replace with actual institution name
\def\institutionCountry {Nederland} % replace with actual institution location
\def\institutionCity    {Utrecht} % replace with actual institution location
\def\dueDate            {30-01-2026} % replace with actual due date

\def\editorsVersion     {false} % true om editorsnote e.d. te tonen, false voor verbergen
\def\makeTitlePage      {true}  % true om titelpagina te maken, false om over te slaan
\def\makeTOCpage        {true}  % true om inhoudsopgave te maken, false om over te slaan
\def\makeBibliography   {true}  % true om bibliografie te maken, false om over te slaan

\def\listMark           {-} % itemize marker, e.g. '-', '*', '\textbullet', etc.
% -----------------------------------------
\begin{document}


    \newpage~
    \newpage

    \begin{figure}[H]
        \centering

            \begin{figure}[H]
                \centering
                \includegraphics[width=\columnwidth]{images/Afbeelding1}
            \end{figure}

            \begin{figure}[H]
                \centering
                \includegraphics[width=\columnwidth]{images/Afbeelding2}
            \end{figure}

        \caption{Boven de visualisatie van de NOS, Onder de visualisatie van mij.}
        \label{fig:info-uitwisseling-1}
    \end{figure}

    \begin{figure}[H]
        \centering
        \includegraphics[height=5cm]{images/colorpallete1_colorblind}
        \caption{Kleurenpalet voor verschillende kleurenzienstoornissen}
        \label{fig:colorpallete1_colorblind}
    \end{figure}

    \newpage

    \section{Bezoekersgedrag Coronamaatregelen}

    \parencite{Ben21}

    \lorem
    \lorem
    \lorem

    \newpage

    \begin{figure}[H]
        \centering

            \begin{figure}[H]
                \centering
                \includegraphics[width=\columnwidth]{images/Afbeelding3}
            \end{figure}

            \begin{figure}[H]
                \centering
                \includegraphics[width=\columnwidth]{images/Afbeelding4}
            \end{figure}

        \caption{Boven de visualisatie van het NRC, Onder de visualisatie van mij.}
        \label{fig:info-uitwisseling-2}
    \end{figure}

    \begin{figure}[H]
        \centering
        \includegraphics[height=5cm]{images/colorpallete2_colorblind}
        \caption{Kleurenpalet voor verschillende kleurenzienstoornissen}
        \label{fig:colorpallete2_colorblind}
    \end{figure}

    \newpage

    \section{Leesvaardigheidsscores Nederlandse scholieren}

    Deze bron heb ik gevonden op het NRC, een landelijke krant die hier verslag deed over de dalende leesvaardigheidsscores van de Nederlandse scholieren. \parencite{Cla23}
    Ik heb hem gevonden door te zoeken op onderwerpen die vaak grafisch worden weergegeven, zoals onderwijsresultaten.
    \newline\newline
    Bij de linker (originele) grafiek is gekozen om aan te geven dat de dramatische keldering heeft gemaakt, echter is dit wel vertekend, omdat de grafiek niet op 0 begint, maar op 450.
    Dit is, als je er snel een blik op werpt zoals je in een krantenartikel vaak doet, erg \textbf{misleidend}.
    Niet alleen is de Y-as niet helemaal geweldig gekozen, ook op de X-as schaadt er wat.
    Hier is op het cruciaalste punt (vanaf het moment van schrijven) namelijk vorig jaar een gat van 4 in plaats van 3 jaar gemaakt.
    Dit wordt echter niet goed visueel gecommuniceerd, door bijvoorbeeld de X-as tussen die punten een jaar breder te maken.
    Zo kun je erg lastig redeneren of er tussen 2015 en 2018 dezelfde jaarlijkse daling is ingezet als tussen 2018 en 2022.
    \newline\newline
    Als ik naar de grafiek kijken (en het bijbehordende artikel lees) dan is het vooral de bedoeling om de lezer te laten zien dat Nederland extreem slecht scoort in vergelijking met andere landen/regio’s.
    Daarom heb ik in mijn verbeterde grafiek de focus gelegd op het vergelijken van de verschillende landen/regio’s.
    Ik heb hier gekozen voor een barchart, omdat dit het makkelijkst is om verschillende landen met elkaar te vergelijken.
    De landen/regio’s zijn op laag naar hoog in het eerste meetjaar (2003) gesorteerd, en deze volgorde is vastgehouden door de vervolgende jaren heen.
    Zo kun je makkelijk zien hoe landen/regio’s zich ten opzichte van elkaar hebben ontwikkeld.

    Ook heb ik hier gekozen voor een kleurenschema dat goed bestand is tegen verschillende soorten kleurenblindheid.
    Er is getest op Protan- Deuter- en Tritanopie en bij alle soorten bleven de kleuren goed te onderscheiden van elkaar.
    Daarnaast is de Y-as hier wel op 0 begonnen, zodat de verschillen in scores eerlijker worden weergegeven.


    \newpage

    \begin{figure}[H]
        \centering

            \begin{figure}[H]
                \centering
                \includegraphics[width=\columnwidth]{images/Afbeelding5}
            \end{figure}

            \begin{figure}[H]
                \centering
                \includegraphics[width=\columnwidth]{images/Afbeelding6}
            \end{figure}

        \caption{Boven de visualisatie van Oost, Onder de visualisatie van mij.}
        \label{fig:info-uitwisseling-3}
    \end{figure}

    \begin{figure}[H]
        \centering
        \includegraphics[height=5cm]{images/colorpallete3_colorblind}
        \caption{Kleurenpalet voor verschillende kleurenzienstoornissen}
        \label{fig:colorpallete3_colorblind}
    \end{figure}

    \newpage

    \section{Zetelverdeling Tweede Kamer 2021 vs 2023}

    De linker grafiek (tevens de originele grafiek) heb ik gevonden op de website van Oost, die verslag doet over voornamelijk Overijssel. \parencite{Oos23}

    In de linker grafiek gebeurt heel veel.
    Je ziet een momentopname van 2021 en 2023 van de zetelverdeling van de Tweede Kamer.
    Maar ook de winst en verlies tussen de twee jaren.
    Deze grafiek gebruikt oppervlakte, ronding en ook verbergen ze de onzekerheid aangezien het om een prognose en 6\% van de stemmen nog niet zijn meegenomen in de visualisatie.
    Verder is de data-inkt ratio toch tamelijk laag, omdat er erg veel poespas gebruikt is om hetzelfde aan te geven.

    Om deze grafiek te verbeteren heb ik in dit geval heb ik gekozen voor een barchart waarbij de start en eindposities de start en eindwaardes van de hoeveelheid zetels die wordt (verwacht) behaald te worden.
    Ik heb hier gekozen voor een dubbele encodering.
    Als eerste de kleur van de balken, in dit geval het rood/groen kleurenschema.
    Deze is wel slechter te zien door mensen met Protanomalie of Protanopie, maar aangezien er tegelijkertijd gebruik wordt gemaakt van de richting van de balk om aan te geven of er een positieve of negatieve verschuiving in het behaalde aantal zetels is, kan deze grafiek ook makkelijk door mensen met (een vorm van) kleurenblindheid bekeken worden.

    Ik heb alleen helaas niet kunnen vinden wat de onzekerheid van de grafiek was, aangezien deze nu ook in mijn verbeterde grafiek niet aanwezig is.
    Had ik deze wel gevonden, dan was deze in schuifgestreepte manier terug te vinden aan het einde van de balk in dezelfde kleur als de balk zelf.

    \newpage

    \begin{figure}[H]
        \centering

            \begin{figure}[H]
                \centering
                \includegraphics[width=\columnwidth]{images/Afbeelding7}
            \end{figure}

            \begin{figure}[H]
                \centering
                \includegraphics[width=\columnwidth]{images/Afbeelding8}
            \end{figure}

        \caption{Boven de visualisatie van het CBS, Onder de visualisatie van mij.}
        \label{fig:info-uitwisseling-4}
    \end{figure}

    \begin{figure}[H]
        \centering
        \includegraphics[height=5cm]{images/colorpallete4_colorblind}
        \caption{Kleurenpalet voor verschillende kleurenzienstoornissen}
        \label{fig:colorpallete4_colorblind}
    \end{figure}

    \newpage

    \section{Autokleuren in Nederland}

    De linker grafiek van het CBS \parencite{CBS22} laat de verdeling van autokleuren in Nederland zien.
    Er gebeurt te veel tegelijk.
    De data wordt weergegeven met kleine autootjes die in schuine rijen liggen, met een duidelijk perspectief en een soort 3D-effect.
    Er is duidelijk een visualisatietijger een goede woensdagmiddag mee bezig geweest en het ziet er leuk uit.
    Hierdoor wordt het helaas wel lastiger dan nodig om de data af te lezen.
    Hoeveel scheelt blauw nu precies met wit?

    Daarnaast is de data-ink ratio laag.
    Een groot deel van de gebruikte inkt gaat op aan decoratie: de autovormpjes, de hellende compositie, de achtergrondkleur en de schaduwwerking.
    Al die elementen dragen niets bij aan het beantwoorden van de vraag welke autokleuren dominant zijn in Nederland.
    Ze leiden af en maken het lastiger om patronen te herkennen.
    De grafische integriteit komt daarmee onder druk te staan, omdat de visuele impact niet meer in verhouding staat tot de feitelijke data.

    Om dit te verbeteren heb ik gekozen voor een veel eenvoudigere visualisatie, namelijk een vlakke, overzichtelijke verdeling waarin elke kleur een duidelijk afgebakend vlak krijgt met het bijbehorende percentage.
    In deze grafiek wordt alleen oppervlakte gebruikt, zonder perspectief of diepte.
    De verhoudingen zijn direct zichtbaar en vergelijkbaar.
    Grijs is onmiskenbaar het grootst, gevolgd door zwart, en daarna pas blauw en wit.
    De kleinere categorieën verdwijnen niet, maar eisen ook niet onnodig aandacht op.

    Door alle niet-essentiële elementen weg te laten, is de data-ink ratio aanzienlijk hoger.
    Vrijwel alles wat je ziet, is data.
    Er is geen poespas, geen decoratie en geen visuele ruis.
    Je krijgt zo in één oogopslag inzicht in de verdeling van autokleuren in Nederland.
    Dat maakt deze simpele grafiek niet alleen rustiger, maar vooral eerlijker en effectiever dan de originele.

    Deze grafiek is helaas niet heel goed te bekijken door mensen met een vorm van kleurenblindheid.
    Echter is hiertegen wel te argumenteren dat de mensen met kleurenblindheid waarschijnlijk ook niet heel erg geïnteresseerd zijn in autokleuren, aangezien zij deze zelf ook niet goed kunnen onderscheiden.
    Ook zou het niet logisch zijn om te proberen een kleurenpalet te maken dat voor alle vormen van kleurenblindheid goed te onderscheiden is, want dan wordt het lastiger voor mensen zonder kleurenblindheid om de kleuren te linken aan de daadwerkelijke autokleuren.

    \newpage

    \begin{figure}[H]
        \centering

            \begin{figure}[H]
                \centering
                \includegraphics[width=\columnwidth]{images/Afbeelding9}
            \end{figure}

            \begin{figure}[H]
                \centering
                \includegraphics[width=\columnwidth]{images/Afbeelding10}
            \end{figure}

        \caption{Boven de visualisatie van Trouw, Onder de visualisatie van mij.}
        \label{fig:info-uitwisseling-5}
    \end{figure}

    \begin{figure}[H]
        \centering
        \includegraphics[height=5cm]{images/colorpallete5_colorblind}
        \caption{Kleurenpalet voor verschillende kleurenzienstoornissen}
        \label{fig:colorpallete5_colorblind}
    \end{figure}

    \newpage

    \section{Gezondheidsrisico's bij twintigers}

    \parencite{Alw16} heeft deze kleurrijke grafiek in de Trouw gepubliceerd.
    Hij laat de toename van gezondheidsrisico’s onder twintigers zien, uitgesplitst naar geslacht.

    In de originele grafiek werd gewerkt met twee rijen poppetjes, blauw voor mannen en roze voor vrouwen.
    Elk poppetje had drie bolletjes die de drie gezondheidsrisico’s moesten voorstellen.
    Op het eerste gezicht oogt dat speels, maar zodra je probeert te begrijpen hoeveel iets toeneemt, dan wordt het lastig.
    De bolletjes vertegenwoordigen percentages, maar die percentages zijn niet visueel in verhouding.
    De cirkels zijn met bijvoorbeeld 45.8\% misschien (want cirkels aflezen is lastig…) 10x kleiner dan 6.9 procent.
    Dat is natuurlijk een extreem misleidend beeld.

    Daarnaast zijn de cirkels die wél met elkaar in verhouding zijn ook erg lastig met elkaar te vergelijken, omdat de gele cirkels bij de mannen tussen de twee einde verdubbeld, terwijl ze erg bijna het zelfde uitzien.

    Daarnaast is de data-ink ratio laag: veel inkt gaat naar poppetjes, kleuren en decoratie, terwijl de daadwerkelijke informatie amper preciezer wordt.

    In de verbeterde grafiek heb ik dit volledig omgedraaid.
    Ik heb gekozen voor lijndiagrammen, uitgesplitst per gezondheidsrisico: bloeddrukverlagers, obesitas en overgewicht.
    De tijd staat op de horizontale as, het percentage op de verticale as.
    Mannen en vrouwen worden onderscheiden met kleur, maar delen exact dezelfde schaal.
    Daardoor ontstaat meteen duidelijkheid.
    Ik zie niet alleen dat elk risico toeneemt, maar ook hoe snel, hoe sterk en wie er hoger zit.

    De belangrijkste channel wordt ingezet: positie op een gemeenschappelijke schaal.
    De lijnen maken trends zichtbaar zonder extra uitleg.
    Vooral bij overgewicht springt het verschil tussen mannen en vrouwen er direct uit.
    Dat hoef je niet te interpreteren, dat zie je gewoon.

    Ook hier is de data-ink ratio hoog.
    Er is nauwelijks decoratie, geen symbolen die niets toevoegen, geen visuele grapjes.
    Vrijwel alles wat je ziet, draagt bij aan begrip.


    \newpage

\end{document}