% ----------- Document settings -----------
\documentclass[nonacm, sigconf, balance=true]{acmart}
\include{core/paperSettings} % import custom general paper settings
\include{core/standardCommands} % import custom commands and packages
% -----------------------------------------
\begin{document}

    \clearpage
    \onecolumn

    \section{Webnavigatie}
    \subsection{Web usage mining}
    Dit gaat over het ontdekken van patronen in webdata, zoals gebruikersgedrag op websites.
    Het doel is om inzicht te krijgen in veel bezochte pagina's en websites of veel voorkomende paden van navigatie.
    Zo kan je begrijpen welke taken en behoeften gebruikers hebben, wat gebruiksonvriendelijke elementen zijn en hoe je de website kunt verbeteren (UI/UX of snelhied etc.).
    Het volgende zijn ook voorbeelden:

    \begin{itemize}
        \item Identificeren van advertentielocaties.
        \item Optimaliseren van menu-desing.
        \item Herkennen van bots en frauduleuze activiteiten.
        \item Personaliseren van content en aanbevelingen.
        \item Voorspellen van de volgende actie van een gebruiker
    \end{itemize}


    \subsection{gebruikersdata}
    \textbf{Gebruikersprofielen}: zijn gegevens die door de gebruiker zelf zijn verstrekt, zoals naam, leeftijd, geslacht, locatie en interesses.
    \textbf{Gebruiksdata}: omvat informatie over hoe gebruikers omgaan met een website, zoals bezochte pagina's, klikgedrag, tijd besteed op pagina's en navigatiepaden.

    Deze data kan je op verschillende manieren verzamelen, zoals:
    \begin{itemize}
        \item \textvf{Webserver}: Voornamelijk klikgedrag, bezochte pagina's en tijd op pagina.
        \item \textvf{Webclient}: Data van één gebruiker op verschillende sites, zoals muisbewegingen, scrollgedrag en interacties.
        \item \textvf{Proxyservers}: Data van meerdere gebruikers, zoals bezochte sites en algemene navigatiepatronen.
    \end{itemize}

\end{document}